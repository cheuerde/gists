\documentclass[xcolor=dvipsnames,aspectratio=169]{beamer}
\setbeamertemplate{caption}[numbered]
\usepackage[ngerman,english]{babel}
\usepackage{multirow}
\usepackage{multicol}
\usepackage{colortbl}
\usepackage{amsmath}
%\usepackage[table]{xcolor}
\usepackage[utf8x]{inputenc}
\usepackage{amsmath,amsfonts,amssymb}
%\usepackage{natbib}
%\usepackage[style=authoryear]{biblatex}
%\addbibresource{phd.bib}
%\DefineBibliographyStrings{ngerman}{andothers={et\addabbrvspace al\adddot}} 
\usepackage{rotating}
\usepackage{graphicx}
\usepackage{geometry}
%\geometry{paperwidth=16cm,paperheight=12cm,left=0cm,top=0cm,right=0cm,bottom=0cm,headsep=0cm,nohead,nofoot}
\usepackage{BeamerColor}
\usetheme{CambridgeUS}
\usecolortheme[named=DodgerBlue4]{structure}
\usecolortheme{dolphin}
%\useoutertheme{smoothbars}
\setbeamertemplate{navigation symbols}{}%remove navigation symbols

\defbeamertemplate*{footline}{my infolines theme}
{
  \leavevmode%
  \hbox{%
  \begin{beamercolorbox}[wd=.333333\paperwidth,ht=2.25ex,dp=1ex,center]{author in head/foot}%
    \usebeamerfont{author in head/foot}\insertshortauthor~~\insertshortinstitute
  \end{beamercolorbox}%
  \begin{beamercolorbox}[wd=.333333\paperwidth,ht=2.25ex,dp=1ex,center]{title in head/foot}%
    \usebeamerfont{title in head/foot}\insertshorttitle
  \end{beamercolorbox}%
  \begin{beamercolorbox}[wd=.333333\paperwidth,ht=2.25ex,dp=1ex,right]{date in head/foot}%
    \usebeamerfont{date in head/foot}\insertshortdate{}\hspace*{2em}
    \insertframenumber{} / \inserttotalframenumber\hspace*{2ex} 
  \end{beamercolorbox}}%
  \vskip0pt%
}

%\logo{\includegraphics[width=2cm]{./kopf3.jpg}}
\title[Parallel Computing in R]{Everyday Multithreading}
\subtitle{Parallel computing for genomic evaluations in \textit{R}}
\author{C. Heuer, D. Hinrichs, G. Thaller}
\institute[]{Institute of Animal Breeding and Husbandry, Kiel University}
\date{\today}
\begin{document}
\begin{frame}
\titlepage
\end{frame}


\section{Introduction}
\subsection*{}
\begin{frame}{Introduction}
\pause
\begin{itemize}
\item High Dimensional livestock data sets 
\item New computational challenges
\item Paradigm shift in breeding programs \textbf{and} computing
\item From sparse to dense MME (or mixtures)
\item High storage, memory and computing demands
\pause
\item \textbf{Solution}: Making use of available hardware resources by parallel computing
\end{itemize}
\end{frame}



\section{Parallel Computing}
\subsection*{}
\begin{frame}{What parallel computing is:}
\pause

Splitting up a big problem into chunks that are simultaneously being solved by several processing units
\pause
\begin{equation*}
\begin{bmatrix}
  -0.09 \\ 
  -0.65 \\ 
  -0.08 \\ 
  -0.95 \\ 
  -0.02 \\ 
  -0.59 \\ 
  -0.55 \\ 
  0.37 \\ 
  -0.05 \\ 
  -0.16 \\ 
  0.59 \\ 
  -0.55 \\ 
  \end{bmatrix}
+
\begin{bmatrix}
  -0.92 \\ 
  0.41 \\ 
  -0.76 \\ 
  -1.61 \\ 
  1.09 \\ 
  -0.33 \\ 
  0.83 \\ 
  3.21 \\ 
  1.29 \\ 
  -0.82 \\ 
  -1.23 \\ 
  -1.08 \\ 
  \end{bmatrix}
=
\begin{bmatrix}
  -1.01 \\ 
  -0.24 \\ 
  -0.83 \\ 
  -2.56 \\ 
  1.07 \\ 
  -0.92 \\ 
  0.27 \\ 
  3.58 \\ 
  1.23 \\ 
  -0.98 \\ 
  -0.64 \\ 
  -1.64 \\ 
  \end{bmatrix} 
\end{equation*}
\end{frame}

\begin{frame}{What parallel computing is:}
Splitting up a big problem into chunks that are simultaneously being solved by several processing units
\begin{equation*}
\begin{bmatrix}
  -0.09 \\ 
  -0.65 \\ 
  -0.08 \\ 
  \hline
  -0.95 \\ 
  -0.02 \\ 
  -0.59 \\ 
    \hline
  -0.55 \\ 
  0.37 \\ 
  -0.05 \\ 
    \hline
  -0.16 \\ 
  0.59 \\ 
  -0.55 \\ 
  \end{bmatrix}
+
\begin{bmatrix}
  -0.92 \\ 
  0.41 \\ 
  -0.76 \\ 
    \hline
  -1.61 \\ 
  1.09 \\ 
  -0.33 \\ 
    \hline
  0.83 \\ 
  3.21 \\ 
  1.29 \\ 
    \hline
  -0.82 \\ 
  -1.23 \\ 
  -1.08 \\ 
  \end{bmatrix}
=
\begin{bmatrix}
  -1.01 \\ 
  -0.24 \\ 
  -0.83 \\ 
    \hline
  -2.56 \\ 
  1.07 \\ 
  -0.92 \\ 
    \hline
  0.27 \\ 
  3.58 \\ 
  1.23 \\ 
    \hline
  -0.98 \\ 
  -0.64 \\ 
  -1.64 \\ 
  \end{bmatrix} 
\end{equation*}
\end{frame}

\begin{frame}{Parallel Computing Paradigms}
\pause
\begin{enumerate}
\item Most importantly: Single Core parallelism  $\rightarrow$ \textbf{Vectorization}
\item Shared Memory multi-threading $\rightarrow$ \textit{OpenMP}
\item Distributed Memory multi-processing $\rightarrow$ \textit{MPI}
\item GPU-programming $\rightarrow$ \textit{CUDA, OpenCL, Intel Xeon-Phi}
\end{enumerate}
\pause
Efficiency/Scaling:
\begin{itemize}
\item Depends on size of the problem: Overhead
\item The less efficient a single-threaded application, the better the scaling
\pause
\item In general: First single-threaded optimization then parallelization
\end{itemize}
\end{frame}


\section{R-Package}
\subsection*{}
\begin{frame}{R-package \textit{cpgen}}
\pause
Advantages of R:
\begin{enumerate}
\item Very flexible open source interpreter language
\item Easy to use, available on all platforms and widely spread
\end{enumerate}
Drawbacks of R:
\pause
\begin{enumerate}
\item Not designed for big data problems
\item Needs a lot of effort to extend R
\item Strictly single-threaded 
\end{enumerate}
\pause
But: Can be extended and multi-threaded through C/C++/Fortran shared libraries
\end{frame}


\begin{frame}{General Implementation}
\pause
\begin{enumerate}
\item \textit{R} as the basic environment for data preparation and supply
\item Linking C++ to R: \textit{Rcpp} (Eddelbuettel and Francois, 2011)
\item Linear Algebra: \textit{Eigen} $\rightarrow$ Vectorization! (Guennebaud et al., 2010)
\item Eigen + Rcpp + Sparse-Matrix support: \textit{RcppEigen} (Bates and Eddelbuettel, 2013)
\item Parallelization: \textit{OpenMP}
\end{enumerate}
\end{frame}

\begin{frame}{Parallel Computing in \textit{cpgen}}
\pause
\begin{center}
\tiny
\includegraphics[width=.55\textwidth]{cgen_mp.png}
\end{center}
\end{frame}

\begin{frame}{Functionality}
\pause
\begin{enumerate}
\item Single-Site Gibbs Sampler for Mixed Models with arbitrary number of random effects (sparse or dense design matrices)
\item Genomic Prediction Module: Different Methods, Cross Validation
\item GWAS Module: \textit{EMMAX} - highly efficient and very flexible single marker GWAS
\item Tools: Genomic additive and dominance relationships, Crossproducts, Covariance Matrices, \dots
\end{enumerate}
\end{frame}


\begin{frame}{Gibbs Sampler}
Runs models of the following form:
\begin{equation*}
\mathbf{y} = \mathbf{Xb} + \mathbf{Z}_{1}\mathbf{u}_1 + \mathbf{Z}_{2}\mathbf{u}_2 + \mathbf{Z}_{3}\mathbf{u}_3 + ... + \mathbf{Z}_{n}\mathbf{u}_n + \mathbf{e}
\end{equation*}
\begin{itemize}
\item For all $u_k$: $MVN(\mathbf{0},\mathbf{I}\sigma^2_{u_k})$
\item If $u_k$ is assumed to follow some $MVN(\mathbf{0},\mathbf{G}_k\sigma_{k}^2)$:
\end{itemize}
 Design matrix must be constructed as: $ \mathbf{Z}_k = \mathbf{Z}_k \mathbf{G}^{1/2}$, yielding independent effect in $\mathbf{u}_k$ (Waldmann et al., 2008). 
\end{frame}




\begin{frame}{Genomic BLUP}
GBLUP can be accomplished very efficiently (Kang et al., 2008):
\begin{equation*}
\mathbf{y} = \mathbf{Xb} + \mathbf{a} + \mathbf{e} ~~~ \textrm{with: } \mathbf{a} \sim MVN(\mathbf{0},\mathbf{G}\sigma^2_a)
\end{equation*} 
By finding the decomposition: $\mathbf{G=UDU'}$ and premultiplying the model equation by $\mathbf{U'}$ we get:
\begin{equation*}
\mathbf{U'y = U'Xb + U'a + U'e}
\end{equation*} 
with:
\begin{align*}
Var(\mathbf{U'y}) &= \mathbf{U'GU} \sigma^2_a + \mathbf{U'U} \sigma^2_e \\
&=  \mathbf{U'UDU'U}\sigma^2_a + \mathbf{I}\sigma^2_e \\
&= \mathbf{D}\sigma^2_a + \mathbf{I}\sigma^2_e
\end{align*}
\end{frame}



\begin{frame}{GWAS}
\begin{itemize}
\item Single marker regression 
\item Controlling for polygenic background effect through [assumed] covariance structure $\mathbf{V}$ of $\mathbf{y}$ - EMMAX (Kang et al., 2010)
\item Obtaining General Least Squares estimates for marker effects:
\end{itemize}
\begin{equation*}
\hat{\mathbf{\beta}} = (\mathbf{X'V}^{-1}\mathbf{X})^{-1}\mathbf{X'V}^{-1}\mathbf{y}
\end{equation*}
This is equivalent to:
\begin{equation*}
\hat{\mathbf{\beta}} = (\mathbf{X}^{\ast\prime}\mathbf{X}^{\ast})^{-1}\mathbf{X}^{\ast\prime}\mathbf{y}^{\ast} , ~~ \textrm{with   } \mathbf{X}^{\ast} =\mathbf{V}^{-1/2}\mathbf{X}, ~~ \mathbf{y}^{\ast} =\mathbf{V}^{-1/2}\mathbf{y}
\end{equation*}
\end{frame}


\begin{frame}{Parallel Computing in \textit{cpgen}}
What is parallelized:
\begin{enumerate}
\item All crossproduct-like computations
\item Sampling of random effects in Gibbs Sampler (dot product, vector-vector subtraction) $\rightarrow$ Fernando et al., 2014
\item Cross Validation for genomic prediction
\item GWAS
\end{enumerate}

\pause
Number of threads being used can be controlled during runtime
\end{frame}

\section{Parallel Scaling}
\subsection*{}
\begin{frame}{2,000 Observations, 50,000 Markers}
\pause
\begin{center}
%\tiny
\includegraphics[width=.8\textwidth]{para_scaling_eaap.jpeg}
\end{center}

\end{frame}

\begin{frame}{Gibbs Sampler (BRR) - 10,000 Markers}
\pause
\begin{center}
%\tiny
\includegraphics[width=.8\textwidth]{para_scaling_lmm_eaap.jpeg}
\end{center}

\end{frame}


\section{Conclusions}
\subsection*{}
\begin{frame}{Conclusions}
\pause
\begin{itemize}
\item Shared Memory multithreading can decrease computation time significantly
\item Bridges the gap between single threaded applications and heavily parallelized standalone programs for HPC Clusters
\item We can make use of the computational power present in workstation PCs 
$\rightarrow$ \textit{Everyday Multithreading}
\pause
\item \textbf{But}: Size of solvable problem is limited by available memory
\item With 1 TB of memory, the package could fit a BRR model with
3 million observations and 40k markers
\end{itemize}
\end{frame}


\section*{}
\begin{frame}{Acknowledgements}
\begin{columns}[c]
\column{2.3in}
University of Kiel:
\begin{itemize}
\item Dr. Georg Thaller
\item Dr. Teide-Jens Boysen
\item Anita Ehret
\end{itemize}
ISU:
\begin{itemize}
\item Dr. Rohan Fernando
\item Xiaochen Sun
\item Jian Zeng
\item Hao Cheng
\item Dr. Dorian Garrick
\end{itemize}
\column{2.3in}
\includegraphics[width=2.2in]{combined.jpg}
\end{columns}
\end{frame}









\section{}
\begin{frame}
\begin{center}
\huge
Thank you for the attention
\end{center}
\vspace{40pt}
%Package can be obtained from:\\ 
\textbf{Github} \url{https://github.com/cheuerde/cpgen}\\
\textbf{R-Forge} \url{https://r-forge.r-project.org/R/?group_id=1687}
\end{frame}


%%%%%%%
%%%%%%%

\section*{}
\subsection*{}
\begin{frame}[noframenumbering]{Absolute Timings}

10 Cores, 30,000 markers

\begin{itemize}
\item BRR: 1 million observations, 30k iterations $\sim$ 100 hours 
\item GWAS: 10k observations $\sim$  7 minutes
\item G-Matrix: 10k observations $\sim$ 2 minutes
\end{itemize}
\end{frame}


\end{document}